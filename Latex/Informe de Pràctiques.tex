\documentclass[a4paper,10.5pt]{report}

\usepackage[toc]{appendix}
\renewcommand{\appendixname}{Annexos}
\renewcommand{\appendixtocname}{Annexos}
\renewcommand{\appendixpagename}{Annexos}
\usepackage[catalan]{babel}
\usepackage{booktabs}
\usepackage{listings}
\lstdefinestyle{mystyle}{
	language=Python,                 % Lenguaje del código
	frame=single,                    % Marco alrededor del código
	basicstyle=\ttfamily\footnotesize, % Fuente del código
	keywordstyle=\color{blue},       % Color para palabras clave (def, if, else...)
	commentstyle=\color{gray},       % Color para comentarios
	stringstyle=\color{teal},        % Color para strings
	numbers=left,                    % Numeración de líneas
	numberstyle=\tiny\color{gray},   % Estilo de los números de línea
	stepnumber=1,                     % Cada cuántas líneas numerar
	showspaces=false,                 % No mostrar espacios en blanco
	showstringspaces=false,            % No mostrar espacios en strings
	breaklines=true,                   % Romper líneas largas automáticamente
	tabsize=4,                         % Tamaño de tabulación
	captionpos=b,                      % Posición de la caption (arriba o abajo)
	morekeywords={self, as, in},       % Palabras clave adicionales
}
\lstset{style=mystyle} % Aplicar este estilo a todos los listados
\usepackage{tcolorbox}
\usepackage{parskip}
\usepackage{multirow}
\usepackage{graphicx}
\usepackage{booktabs}
\usepackage{caption}
\captionsetup{labelfont=bf}
\usepackage{hyperref}
\usepackage[arrowdel]{physics}
\usepackage[left=1.95cm, right=1.95cm, top=20mm, bottom=20mm]{geometry} 
\usepackage{fancyhdr}
\usepackage{braket}
\usepackage{amsmath, amssymb, amsfonts}
\usepackage{subcaption}
\usepackage{cancel}
\usepackage{float}
\usepackage{titling}
\usepackage{etoolbox}

%Definim el següent entorn per tal de poder posar abstracts a cada capítol.
\newenvironment{chapterabstract}{
	\begin{center}
		\bfseries Abstract
	\end{center}
	\quotation
}{\endquotation}

\usepackage{titlesec}
\usepackage{tikz}


\titleformat{\chapter}[display]{\normalfont\huge\bfseries}{Pràctica \thechapter}{0pt}{\vspace{0.2cm}\huge\bfseries\raggedright}

\title{\textbf{\huge{Informes de Pràctiques. \\ \vspace{0.2cm} Laboratori d'Electromagnetisme}}}
\author{Grup A1}
\date{\today}

\begin{document}
	
\begin{titlepage}
	\centering
	{\LARGE Laboratori d'Electromagnetisme \par}
	\vspace{2cm}
	{\Huge \textbf{Informes de Pràctiques} \par}
	\vspace{3cm}
	{\Large Grup A1 \par}
	\vspace{0.5cm}
	{\Large 1549086: Bujones Umbert, Jun Shan\\1669619: Rama Ariza, Raul\\  1672980: González Barea, Eric\\1644841: Vilarrúbias Morral, Natàlia \par}
	\vspace{2cm}
	{\Large Març $-$ Maig 2025 \par}
	\vspace{2cm}
	
	\begin{figure}[h]
		\centering
		\includegraphics[width=0.3\linewidth]{screenshot001}
		\label{fig:screenshot001}
	\end{figure}
\end{titlepage}

\tableofcontents
\newpage

\chapter{Representació de camps} 

\begin{chapterabstract}
	En aquesta pràctica estudiem diferents problemes electrostàtics en medis conductors aprofitant la dualitat existent entre la densitat de corrent $\vec{J}$ i el vector desplaçament $\vec{D}$. El nostre objectiu és trobar experimentalment les superfícies equipotencials per a determinades geometries, amb una simetria tal que podem reduir el problema a dues dimensions espacials. Una de les distribucions de càrrega amb què treballem és un condensador de plaques planoparal·leles ideal; per aquest cas, a més a més, fem el càlcul de la seva capacitat per unitat de longitud, partint del teorema de Gauss.
\end{chapterabstract}

\section{Introducció i fonament teòric}
Per a materials lineals, isòtrops i homogenis, sota la presència d'un camp electrostàtic $\vec{E}$, s'apliquen les següents equacions si el medi és conductor:
\begin{align}
	\vec{\nabla} \cross \vec{E} = 0 \\
	\vec{J} = \sigma \vec{E} \label{eq1.2}  \\ 
	\vec{\nabla}\cdot \vec{J} = 0 \label{eq1.3}
\end{align}
O, si el medi és dielèctric:
\begin{align}
	\vec{\nabla} \cross \vec{E} = 0 \\
	\vec{D} = \varepsilon \vec{E} \label{eq1.5} \\ 
	\vec{\nabla}\cdot \vec{D} = 0 \label{eq1.6}
\end{align}

Per aquest tipus de medis $\varepsilon$ i $\sigma$ són constants, així, combinant les darreres equacions trobem:
\begin{align}
	\vec{\nabla} \cross \vec{J} = 0 \\
	\vec{\nabla} \cross \vec{D} = 0
\end{align}
Això últim implica que tant $\vec{J}$ com $\vec{D}$ són camps conservatius i, per tant, es poden definir com el gradient (canviat de signe) d'una funció escalar, és a dir:
\begin{align}
	\vec{J} = -\vec{\nabla}{U} \\
	\vec{D} = -\vec{\nabla}{U'}
\end{align}
A partir de les equacions \eqref{eq1.3} i \eqref{eq1.6} podem deduir
\begin{equation}
	\nabla^2 U  = 0, \hspace{0.25cm} \nabla^2 U' = 0 \label{eq1.11}
\end{equation}
que són les corresponents equacions de Laplace. Així, donats uns potencials escalars $U$ i $U'$ que satisfacin les condicions de contorn i \eqref{eq1.11}, podem trobar $\vec{J}$ i $\vec{D}$, respectivament.

Si comparem les equacions \eqref{eq1.2} i \eqref{eq1.5} per una banda i les equacions \eqref{eq1.3} i \eqref{eq1.6}, podem veure que qualsevol solució per $\vec{J}$ és també una solució vàlida per $\vec{D}$, sempre que estiguem sota condicions de contorn equivalents i sempre que ni $\sigma$ ni $\vec{E}$ presentin discontinuïtats. Per tant, si coneixem una solució per un medi conductor, podrem trobar-ne una equivalent pel medi dielèctric intercanviant $\varepsilon$ per $\sigma$.

Recordem que, per poder calcular la capacitat per unitat de longitud d'un condensador de plaques planoparal·leles, considerem una superfície equipotencial que tanca una de les plaques del condensador. En virtut del teorema de Gauss tenim que:
\begin{equation}
	q = \varepsilon \int_S \vec{E}\cdot \vec{n}dS
\end{equation}
Assumint que el condensador és infinitament llarg en la direcció $z$, podem assegurar que el camp és constant en aquesta direcció i, per tant, $dS = Zdl$, on $dl$ és el diferencial de longitud a la intersecció de la superfície equipotencial amb un pla perpendicular al condensador. Amb això tenim que:
\begin{equation}
	\frac{q}{Z} = \varepsilon \oint_C E dl \label{eq1.13}
\end{equation}
\section{Metodologia experimental}
\subsection{Representació de corbes equipotencials}
Per tal de poder representar les línies equipotencials usem fulls de paper impregnats amb carbó de resistències compreses en un rang de 5 k$\Omega$ $-$ 20 k$\Omega$ per quadrat, que actuaran com a medis conductors (de conductivitat $\sigma$ homogènia) entre els elèctrodes. Les distribucions de càrrega (elèctrodes) les dibuixem usant un retolador que desprèn una tinta conductora, produïda per partícules de plata en suspensió en un líquid (veure figura \ref{fig1.1a}); per assegurar-nos que la conductivitat d'aquesta tinta esdevé màxima deixem reposar els dibuixos durant 20 minuts, aproximadament. Per evitar possibles problemes de falta de càrregues als elèctrodes, ens hem assegurat de dibuixar línies suficientment gruixudes.

\begin{figure}[h]
	\centering
	\begin{subfigure}{0.45\textwidth}
		\centering
		\includegraphics[width=\linewidth]{screenshot002}
		\caption{Paper conductor i retolador de tinta basada en una suspensió de plata.}
		\label{fig1.1a}
	\end{subfigure}
	\hfill
	\begin{subfigure}{0.45\textwidth}
		\centering
		\includegraphics[width=\linewidth]{screenshot003}
		\caption{Muntatge experimental per a la representació de les corbes equipotencials pels dos fils infinits. D'esquerra a dreta: Font de corrent DC connectada als dos elèctrodes (dos punts pel cas representat) mitjançant dos cables; suro amb el paper conductor en el que prèviament s'han dibuixat els dos elèctrodes, enganxat amb xinxetes; multímetre usat per mesurar les diferències de potencial.}
		\label{fig1.1b}
	\end{subfigure}
	\caption{Paper conductor i muntatge experimental per a la representació de les corbes equipotencials.}
	\label{fig1.1}
\end{figure}

Hem treballat amb les següents 3 distribucions: dues línies verticals (que són la projecció d'un condensador de plaques planoparal·leles), dos punts (projecció de dos fils infinits) i dues línies secants amb un punt entre elles (projecció de dos plans infinits formant angle d'aproximadament 60 º amb un fil infinit entre els dos).

Amb el pretext de generar el camp sobre les distribucions dibuixades, s'ha fixat el paper conductor (en el qual hem fet els dibuixos) sobre un suro usant xinxetes i hem connectat els elèctrodes (per a cada distribució per separat) a una font de corrent continu (DC) usant un parell de cables i més xinxetes. Per mesurar la diferència de potencial hem usat un multímetre, deixant un cable fixat a un dels dos elèctrodes (establint així una referència de potencial) i l'altre lliure per tal de fer mesures de $\Delta V$ a qualsevol altre punt del paper (veure figura \ref{fig1.1b}). Prèviament, però, ens hem assegurat què la diferència de potencial entre dos punts en els conductors (els elèctrodes dibuixats) no fos major de l'1\%\footnote{Recordem que, per ser aquests materials conductors, hem de tenir un potencial constant en tot el seu volum i, en particular, sobre la seva superfície.}.

Per dibuixar les corbes equipotencials usem el cable lliure del multímetre per buscar aquestes corbes sobre el paper. Marquem tots els punts que estan a un mateix potencial amb un llapis i, tot seguit, unim aquests punts amb una línia. Repetint això un seguit de cops podem construir vàries corbes equipotencials. Per tal d'assegurar-nos que es tanquen, és millor que comencem a buscar les corbes des de l'exterior dels nostres elèctrodes. Si comencem per l'interior, com que tindrem una densitat de corbes molt major, serà més fàcil que la línia escollida no s'acabi tancant. Veurem que ens interessa trobar corbes que tanquin els nostre elèctrodes per tal de poder aplicar el teorema de Gauss (en especial pel cas del condensador de plaques planoparal·leles).

Finalment, per representar el camp elèctric $\vec{E}$, hem dibuixat línies que sortien dels elèctrodes i tallaven les corbes equipotencials perpendicularment (en ambdós casos). 

\subsection{Càlcul de la capacitat del condensador}
Si aproximem la integral donada per l'equació \eqref{eq1.13} per un sumatori i calculem el camp $E_i$ segons
\begin{equation}
	E_i \approx \frac{\Delta V_i}{\Delta r_i} \label{eq1.14}
\end{equation}
on $\Delta r_i$ és la distància radial i $\Delta V_i$ és la diferència de potencial de l'element $\Delta l_i$ tenim:
\begin{equation}
	\frac{Q}{Z} \approx \varepsilon \sum_i \frac{\Delta V_i \Delta l_i}{\Delta r_i}
\end{equation}
Usant que la capacitat d'un condensador de plaques planoparal·leles es correspon amb 
\begin{equation}
	C = \frac{Q}{\Delta V}
\end{equation}
tenim que la capacitat per unitat de longitud, sota les aproximacions usades és:
\begin{equation}
	\frac{C}{Z} = \frac{Q/Z}{\Delta V} = \frac{\varepsilon}{\Delta V} \sum_i \frac{\Delta V_i \Delta l_i}{\Delta r_i} \label{eq1.17}
\end{equation}
on $\Delta V$ és la diferència de potencial a la que hem sotmès les dues plaques del condensador.

\subsection{Simulacions}
Els codis de les simulacions es poden trobar a l'annex \ref{an:a4}.

Per la simulació del condensador planoparal·lel, considerem la projecció com la contribució de 200 carregues puntuals (totes del mateix valor, pel fil esquerra de valor $-q$, pel dret $q$) distribuïdes uniformement a través de cada fil. Un fil es troba a $x = d/2$, i el segon a $x = - d/2$. Les carregues van des de $y = -H/2$ fins $y = H/2$. Per a cada una de les càrregues, calculem la seva contribució al potencial i al camp elèctric en tots els punts de la malla $(X, Y)$, que ho fem usant el principi de superposició.

Per a la simulació dels dos fils infinits paral·lels, considerem la seva projecció com dues càrregues puntuals separades una distància $(d+2r)$, on aquest valor correspon a la separació entre els centres dels discos dibuixats experimentalment.

Definim les funcions que descriuen el potencial i el camp elèctric, que corresponen a les expressions analítiques d'aquests magnituds per a una càrrega puntual. Es calcula la contribució de cada càrrega en tots els punts de la malla i  apliquem el principi de superposició per obtenir el camp elèctric i el potencial total en cada punt de l’espai.

Per últim ens caldrà simular la distribució extra, que hem proposat amb l'objectiu de comprovar el mètode de les imatges, que son els dos plans secants infinits en la direcció $Z$ amb un cert angle entre ells (uns 60º), on entre ells passa un fil infinit. Si posem l'origen de coordenades en el punt on s'intersequen els plans, i definim $d$ com la distancia en l'eix $x$ des de l'origen al fil infinit, pel mètode de les imatges concloem que, per complir les nostres condicions de contorn, que es que sobre els plans el potencial sigui el mateix, hem de col·locar un total de $2\pi/\Delta\theta-1$ càrregues imatge (on $\Delta\theta$ es l'angle entre els plans), totes a una distància $d$ de l'origen i amb un angle entre elles de $\Delta\theta$ formant així una distribució simètrica de càrregues en forma de cercle \ref{fig:1.5}. 
Per fer la simulació d'aquesta distribució nomes hem de d'aplicar el principi de superposició dels camps i potencials que generarien cadascuna de les càrregues puntuals a la posició que hem descrit.

També ens serà interessant mes endavant tenir un càlcul explicit de la nostra distribució en concret, ja que el mètode de les imatges suposa plaques infinites per establir les condicions de contorn. Així que també obtindrem les expressions analítiques, els detalls dels quals es donen als annexos \ref{an:a6}. 
\section{Resultats i discussió}
Els resultats experimentals es poden trobar a l'annex \ref{an:a2}.

\subsection{Condensador de plaques planoparal·leles}
La primera configuració estudiada correspon a un condensador planoparal·lel, format per dos plans infinits (en l'eix $Z$) carregats i separats per una distància $d$. En projectar aquests plans sobre el pla $XY$, es representen com dos fils finits paral·lels separats també per una distància $d$. 

Les corbes equipotencials trobades experimentalment són les que es poden observar a la figura \ref{fig:1.2a}. Observem que dins la regió entre plaques, les línies equipotencials es mantenen gairebé paral·leles, indicant un camp elèctric uniforme i dirigit perpendicularment a les plaques. 

A les vores de les plaques, es manifesta l’efecte de vora, on el camp deixa de ser uniforme i es distribueix de manera més complexa, amb línies equipotencials corbades cap a l’exterior. Aquest efecte és més acusat en condensadors de mida finita, ja que en el cas ideal de plaques infinites, el camp fora de la regió entre elles seria nul. 

Com ja hem comentat abans podem utilitzar la llei de gauss per calcular la càrrega per unitat de longitud que tindria el condensador infinitament llarg en la direcció $Z$. Per fer-ho utilitzarem l'expressió \eqref{eq1.17} abans deduïda, i les dades experimentals que presentem en l'annex \ref{an:a2}, amb tot això resulta:

\begin{equation}
	\frac{C}{Z} = \frac{Q/Z}{\Delta V} = \frac{\varepsilon}{\Delta V} \sum_i \frac{\Delta V_i \Delta l_i}{\Delta r_i} = (2,06 \pm 0,21)\varepsilon \hspace{0,2cm} (F/m)
\end{equation}

Per comparar amb lo que seria un condensador ideal fem el càlcul suposant que el nostre cas ho és. Per un condensador ideal tenim que:

\begin{equation}
	\frac{C}{Z} = \frac{A\varepsilon/d}{Z} = \frac{H}{d}\varepsilon = \frac{7,961cm}{5,598cm}\varepsilon = (1,422 \pm 0,001)\varepsilon \hspace{0,2cm} (F/m)
\end{equation}
On $A$ es l'area de la placa, i $H$ l'altura d'aquesta.
Podem notar ràpidament que el nostre condensador es desvia notablement de lo que es un condensador ideal, no només això, sinó que te una capacitat per unitat de longitud superior, cosa que no pot ser. Hi ha principalment dos motius pels quals no hem obtingut la càrrega per unitat de longitud real, i es que experimentalment hauríem de haver fet més punts per tenir unes línies de camp més precises i així poder fer una molt millor aproximació a la integral donada per \eqref{eq1.17}. L'altre font d'error, i la mes notable, es que per que aquesta aproximació sigui lo mes bona possible requereix que l'aproximació donada per \eqref{eq1.14} sigui lo mes acurada possible, i això suposa que les línies equipotencials sobre les que mesurem la diferencia de potencial per saber el camp allà han de ser lo mes pròximes possibles, ja que en veritat:
\begin{equation}
	E = \frac{\partial V}{\partial r}
\end{equation}
On $E$ es el modul del camp, i $\partial r$ es el diferencial de longitud en la direcció del camp (i.e. perpendicular a les línies equipotencials), per lo que si mesurem un  $\Delta r$ haurà de ser petit, cosa que veient les nostres línies experimentals de potencial no es cert. El fet de no fer línies properes no solament fa que no sigui una bona aproximació a la derivada del potencial, ja que el camp varia molt en el nostre $\Delta r$, sinó que, com hem dit $E = \frac{\partial V}{\partial r}$ es cert sempre que $\partial r$ sigui en la direcció del camp, ja hem tractat de ser conscients d'això en mesurar $\Delta r$ perpendicular a la línia equipotencial, però en haver tanta distància entre les dues línies entre elles el camp es corba bastant per lo que es perd el fet de mesurar $\Delta r$ paral·lel a $\vec{E}$, especialment fora del condensador. 

\begin{figure}
	\centering
	\begin{subfigure}{0.45\linewidth}
		\centering
		\includegraphics[height=5.5cm]{dibplaques} % Augmenta la mida
		\caption{Corbes equipotencials (en blanc) i línies de camp $\vec{E}$ (en lila) pel cas del condensador planoparal·lel trobades experimentalment.}
		\label{fig:1.2a}
	\end{subfigure}
	\hfill
	\begin{subfigure}{0.53\linewidth}
		\centering
		\includegraphics[height=6.5cm]{figplaques1} % Mateixa alçada
		\caption{Simulació del camp elèctric i del potencial pel condensador planoparal·lel.}
		\label{fig:1.2b}
	\end{subfigure}
	\caption{Representació i simulació de les corbes equipotencials i línies de camp per la distribució del condensador planoparal·lel.}
	\label{fig:1.2}
\end{figure}


\subsection{Fils infinits}
La segona configuració estudiada és la projecció en el pla $XY$ (pla de la imatge) de dos fils infinits separats per una distància constant. 

Les corbes equipotencials trobades experimentalment són les que es poden observar a la figura \ref{fig:1.3a}. Observem com aquestes formen el·lipses on els fils estan cada cop més descentrats conforme agafem corbes més externes. 
\begin{figure}[h]
	\centering
	\begin{subfigure}{0.45\linewidth}
		\centering
		\includegraphics[width=\linewidth]{screenshot004}
		\caption{Corbes equipotencials (en blanc) i línies de camp $\vec{E}$ (en lila) pel cas dels dos fils infinits trobades experimentalment.}
		\label{fig:1.3a}
	\end{subfigure}
	\hfill
	\begin{subfigure}{0.5\linewidth}
		\centering
		\includegraphics[width=\linewidth]{figfils}
		\caption{Simulació del camp elèctric i del potencial pels dos fils infinits.}
		\label{fig:1.3b}
	\end{subfigure}
	\caption{Representació i simulació de les corbes equipotencials i línies de camp per la distribució de dos fils infinits.}
	\label{fig:1.3}
\end{figure}

Els resultats teòrics indiquen que, en un sistema de dos fils infinits, les corbes equipotencials vénen donades per la següent equació:
\begin{equation}
	y^2+\left( x+a\frac{1+k^2}{1-k^2}\right)^2 = a^2\left( \frac{2k}{1-k^2}\right)^2  \label{eqsuppon}
\end{equation}
Que és l'equació d'una circumferència, de manera que, teòricament, les corbes equipotencials haurien de ser circumferències de radi $R = a\frac{2k}{1-k^2}$ que tenen el seu centre lleugerament desplaçat en l'eix de les $x$ per un factor $a\frac{1+k^2}{1-k^2}$. Notem que tant $a$ com $k$ són dues constants\footnote{Els valors de les constants i la demostració d'aquest resultat es pot trobar a l'annex \ref{an:a1}.}. Es pot veure que això és exactament així si ens atenim als resultats obtinguts a la simulació de la figura \ref{fig:1.2b}.

El fet que els nostres resultats es desviïn del predit per la teoria és degut als diferents errors experimentals comesos durant l'evolució de la pràctica: S'assumeix que el paper conductor té una conductivitat $\sigma$ homogènia, però això no necessàriament ha de ser així (pot ser que presenti inhomogeneïtats), de forma que els resultats de l'experiment es poden veure alterats; els punts que representen la projecció dels dos fils infinits en el pla no són perfectament rodons, ja que han sigut dibuixats a ma (amb les imprecisions que això comporta); les corbes equipotencials i les línies de camp es poden veure afectades per l'efecte punta en les vores dels materials carregats, on s'acumula més densitat de càrrega. 


\subsection{Fil infinit i dos plans}

La tercera configuració estudiada i proposada per nosaltres és la projecció en el pla $XY$ de dos plans infinits (en l'eix $Z$) formant un angle de $60^o$, més un fil infinit (en l'eix $Z$) dins la regió entre els plans.

Les corbes equipotencials trobades experimentalment són les que es poden observar a la figura \ref{fig:1.4a}.

\begin{figure}[h]
	\centering
	\begin{subfigure}{0.49\linewidth}
		\centering
		\includegraphics[width=\linewidth]{confinventJS}
		\caption{Corbes equipotencials (en blanc) i línies de camp $\vec{E}$ (en lila) pel cas dels dos plans infinits i un fil infinit trobades experimentalment.}
		\label{fig:1.4a}
	\end{subfigure}
	\hfill
	\begin{subfigure}{0.49\linewidth}
		\centering
		\includegraphics[width=\linewidth]{figplacarara}
		\caption{\textcolor{red}{cambiar estilo del grafico para hacerlos todos igual} Simulació del camp elèctric i del potencial pels dos plans i el fil infinit.}
		\label{fig:1.4b}
	\end{subfigure}
	\caption{Representació i simulació de les corbes equipotencials i línies de camp per la distribució de dos plans infinits amb un fil infinit enmig.}
	\label{fig:1.4}
\end{figure}

Ja notem com aquesta configuració de càrrega es molt més complexa que les anteriors. Tenim que una de les distribucions te forma de triangle, i es molt notable el efecte punta que presenta en cada un dels extrems d'aquest, es pot veure clarament com d'aquí el camp divergeix mes que en altres punts.

\begin{figure}[h]
	\centering
	\includegraphics[width=0.49\linewidth]{figV2imagenes}
	\caption{Representació i simulació de les corbes equipotencials i línies de camp per la distribució de dos plans infinits amb un fil infinit enmig.}
	\label{fig:1.5}
\end{figure}
\textcolor{red}{no entiendo pq esta figura (la 1.5) se pone abajo de la sección}

Hem fet la simulació pel mètode de les imatges que presentem a la figura \ref{fig:1.5}. Aquesta presenta un comportament molt similar a l'obtingut experimentalment, a la regió entre la càrrega i les plaques es pràcticament idèntic, però després, conforme ens allunyem de la càrrega el comportament es lleugerament diferent, per exemple, sobre les proximitats de l'eix $x$ llunyans al fil el camp apunta cap a fora de la distribució (tant experimentalment com en la simulació analítica), metre que en la simulació pel mètode de les imatges es al reves. Es d'aquestes diferencies que es nota present el fet que el mètode de les imatges suposa unes plaques infinites, per lo que en en $x$ grans predomina encara el comportament de les plaques que es repulsiu, metre que a la realitat no es així. Per això es important tenir present on el mètode de les imatges reprodueix be les condicions de contorn, ja que només allà el comportament serà comparable amb la realitat física. 

Una cosa que no vam arribar a notar experimentalment per falta de línies en aquesta zona però que es molt interessant, es que sobre l'eix $x$ mes enllà del fil infinit existeix un punt pel qual el camp es nul, i es per tant un punt d'equilibri. Era esperable que existeixi un punt tal que la distribució de càrrega de les plaques anul·lés la del fil, però es interessant veure-ho a sa simulació, i es que aquest punt es també una clara diferencia entre el mètode de les imatges i el comportament analític i real, es un punt en que canvia el comportament atractiu-repulsiu del sistema, i un límit per on el mètode de les imatges deixa de ser vàlid per no reproduir correctament les condicions de contorn donat per la llargada de les plaques. \\

\section{Conclusions}
Aquesta pràctica s'ha basat en l'ús d'un multímetre per tal de mesurar el potencial elèctric a diferents distàncies de les distribucions de càrrega dibuixades. Això ens ha permès, per una banda, traçar línies equipotencials i, per altra, obtenir un valor de la capacitat d'un condensador.

Pel que fa a les línies equipotencials, aquestes ens han servit de guia per dibuixar, posteriorment, les respectives línies de camp elèctric, ja que, com sabem de teoria, aquestes són perpendiculars a les equipotencials. A més a més, per a les tres geometries considerades, hem generat les corresponents simulacions. Així podem comparar els resultats experimentals amb els teòrics amb més facilitat.

Tal i com s'ha anat mencionant, tant les línies equipotencials com les de camp basades en les mesures del multímetre (resultats experimentals) segueixen la mateixa tendència que les simulacions. Tot i no coincidir-hi exactament a causa de l'efecte punta i els possibles errors aleatoris, la geometria i el sentit de les línies de camp són coherents.

CÀLCUL CAPACITAAAAAAAAAAAAAT

Així, a partir de tres distribucions de conductors diferents amb suficient simetria com per reduir el problema a dues dimensions, hem comprovat l'existència de la dualitat entre la densitat de corrent $\vec{J}$ i el vector desplaçament $\vec{D}$.

\chapter{Força entre corrents}
\begin{chapterabstract}
	En aquesta pràctica mesurem la força entre dos fils pels quals hi circula un corrent elèctric, comprovant que la llei de Biot i Savart se satisfà, via diferents metodologies. Amb aquests resultats fem una estimació de la permeabilitat magnètica $\mu_0$, tot comparant-la amb el valor teòric. A més a més, utilitzant el mateix sistema, mesurem la component horitzontal del camp magnètic terrestre al laboratori.
\end{chapterabstract}
\section{Introducció i fonament teòric}
Per quantificar la interacció magnètica entre dos circuits arbitraris tancats pels quals hi circula un corrent constant, és a dir, per mesurar la força que fa un circuit sobre l'altre podem usar que:
\begin{equation}
	\vec{F}_1 = \frac{\mu_0}{4\pi}I_1I_2\oint \oint \frac{\mathrm{d}\vec{l}_1\cross[\mathrm{d}\vec{l}_2 \cross (\vec{r}_1-\vec{r}_2)]}{\abs{\vec{r}_1 - \vec{r}_2}^3}\label{eq:2.1}
\end{equation}
on $\mu_0$, $I_1$ i $I_2$ són les intensitats dels circuits 1 i 2, respectivament, d$\vec{l}_1$ i d$\vec{l}_2$ són els elements infinitesimals de línia i $\vec{r}_1$ i $\vec{r}_2$ són les respectives posicions d'aquests elements.

Per tal de simplificar la integral anterior definim el camp d'inducció magnètica (o densitat de flux magnètic) $\vec{B}(\vec{r})$ en un punt arbitrari $\vec{r}$ com:
\begin{equation}
	\vec{B}(\vec{r}) = \frac{\mu_0}{4\pi} \int_V \vec{J}(\vec{r}_1)\cross\frac{\vec{r}-\vec{r}'}{\abs{\vec{r}-\vec{r}'}^3}\mathrm{d}^3r'
\end{equation} 
Si usem això sobre l'equació \eqref{eq:2.1} podem escriure la força que rep el circuit amb intensitat I degut a la presència del camp d'inducció $\vec{B}$ creat per l'altre circuit segons:
\begin{equation}
	\vec{F} = I \oint \mathrm{d}\vec{l}\cross\vec{B}(\vec{r})
\end{equation}
Emprant una de les equacions pel camp $\vec{B}$
\begin{equation}
	\vec{\nabla}\cross \vec{B} = \mu_0 \vec{J}
\end{equation}
i aplicant el teorema de Stokes
\begin{equation}
	\oint_C \vec{A} \cdot \mathrm{d}\vec{r} = \int_S (\vec{\nabla}\cross \vec{A})\cdot \mathrm{d}\vec{S}
\end{equation}
podem deduir fàcilment el teorema d'Ampère, que ens diu que:
\begin{equation}
	\oint_C\vec{B}\cdot\mathrm{d}\vec{l} = \mu_0\int_S\vec{J}(\vec{r})\cdot\vec{n}\mathrm{d}S
\end{equation}
d'on, mitjançant un càlcul ràpid podem trobar el camp d'inducció magnètica generat per un fil infinit amb intensitat $I$ a una distància $r$ del seu centre:
\begin{equation}
	B = \frac{\mu_0 I}{2\pi r}
\end{equation}
Així doncs, la força, en mòdul que patirà un altre fil (infinit) paral·lel, de longitud $L$ i que és perpendicular al camp $\vec{B}$ serà:
\begin{equation}
	F = \frac{\mu_0 I^2L}{2\pi r}
\end{equation} 
\section{Metodologia experimental}

\subsection{Equilibrat de la balança}
\subsection{Força vs. corrent}
\subsection{Força vs. separació}
\subsection{Mesura del camp magnètic terrestre}

\section{Resultats i discussió}

\subsection{Valor experimental de la permeabilitat magnètica}
\subsection{Camp magnètic terrestre}

\section{Conclusions}

\begin{thebibliography}{99}
	\bibitem{ref1}
	\textit{Col·lecció de problemes de l'assignatura d'Electromagnetisme.}
\end{thebibliography}




\newpage
\begin{appendices}
%SI TOQUEU AIXÒ ELS ANNEXOS PETEN.

\textbf{\Huge{Annexos}}
\renewcommand{\thesection}{\Alph{section}} % Cambia la numeración de capítulos a letras
\renewcommand{\theequation}{\thesection.\arabic{equation}} % Cambia numeración de ecuaciones
\setcounter{equation}{0} % Reinicia contador de ecuaciones en cada sección
\section{Pràctica 1. Representació de camps}
\subsection{Deducció de l'equació \eqref{eqsuppon}}
\label{an:a1}
Per deduir l'expressió donada per l'equació \eqref{eqsuppon} ens basarem ens els resultats del problema 2.18 de la llista de problemes de l'assignatura \cite{ref1}. Suposem dos fils infinits rectilinis i carregats amb densitat de càrrega uniforme $\lambda$ i -$\lambda$ separats per una distància $d=2a$. Siguin $\rho_1$ i $\rho_2$ les distàncies radials de cada fil al punt en el qual volem calcular el camp i el potencial. Per la simetria del sistema podem assegurar que $\vec{E}(\vec{r}) = E(\rho) \vec{e}_{\rho}$, de forma que podem aplicar el teorema de Gauss com se segueix:

\[
\left.
\begin{array}{c}
	\oint \vec{E} \cdot \mathrm{d}\vec{S} = E 2\pi \rho L \\[10pt]
	\frac{Q_{int}}{\varepsilon_0} = \frac{1}{\varepsilon_0} \int \lambda \, \mathrm{d}l = \frac{\lambda}{\varepsilon_0} L
\end{array}
\right\}
\quad \Rightarrow \quad 
\vec{E} = \frac{\lambda}{2\pi \varepsilon_0} \frac{1}{\rho} \vec{e}_{\rho}
\]

Així, essent $\vec{E_1}$ el camp associat al fil amb densitat $\lambda$ i $\vec{E_2}$ el camp associat al fil amb densitat -$\lambda$, tenim:
\begin{align}
	\vec{E_1} & = \frac{\lambda}{2\pi \varepsilon_0} \frac{1}{\rho_1} \vec{e}_{\rho_1} \\
	\vec{E_2} & = -\frac{\lambda}{2\pi \varepsilon_0} \frac{1}{\rho_2} \hat{e}_{\rho_2}
\end{align}
on:
\begin{align}
	\rho_1 & = \sqrt{(x-a)^2+y^2} \\
	\rho_2 & = \sqrt{(x+a)^2+y^2} 
\end{align}

Si calculem el camp usant que
\begin{equation}
	\phi(r) = -\int_{\vec{r}_{ref}}^{\vec{r}}\vec{E}(\vec{r})\cdot \mathrm{d}\vec{r} 
\end{equation}
i aplicant el principi de superposició, és a dir
\begin{equation}
	\phi = \phi_1+\phi_2
\end{equation}
trobem que, el potencial generat per aquesta distribució de càrrega obeeix la següent equació:
\begin{equation}
	\phi = \frac{\lambda}{2\pi \varepsilon_0}\ln\sqrt{\frac{(x+a)^2+y^2}{(x-a)^2+y^2}}
\end{equation}
Escollint
\begin{equation}
	k \equiv e^{\frac{\phi2\pi\varepsilon_0}{\lambda}} = \sqrt{\frac{(x+a)^2+y^2}{(x-a)^2+y^2}}
\end{equation}
i reescrivint segons
\begin{equation}
	k^2[(x-a)^2+y^2]=(x+a)^2+y^2
\end{equation}  
podem desenvolupar fins a arribar a
\begin{equation}
	\boxed{y^2+\left( x+a\frac{1+k^2}{1-k^2}\right)^2 = a^2\left( \frac{2k}{1-k^2}\right)^2}
\end{equation}
que és el que estàvem buscant.

\newpage

\subsection{Expressió analítica de la distribució de dos plans secants i un fil}
\label{an:a6}
Per obtenir les expressions analítiques no queda d'altra que aplicar la definició de potencial i integrar. De la definició de potencial:

\begin{equation}
	\phi (x,y) = \frac{1}{4\pi\varepsilon}\int_S\frac{\sigma(x',y')}{\sqrt{(x-x')^2+(y-y')^2}}dA
\end{equation}
on les coordenades prima son les referents a la distribució de carrega i A' tota la regió on hi hagi càrrega. Separarem la distribució en tres parts, la càrrega puntual, la part de la placa inferior i la superior, finalment aplicarem el principi de superposició per conèixer el potencial que crearia la nostra distribució. El potencial que crea la càrrega puntual a la posició $\vec{r_q}$ és:

\begin{equation}
	\phi_q (x,y) = -\frac{1}{4\pi\varepsilon}\frac{q}{\sqrt{(x-x')^2+(y-y')^2}}
\end{equation}
Per resoldre la integral per les plaques hem de tenir en comte que es una distribució unidimensional de càrrega, per lo que $\sigma(x',y') \longmapsto \lambda(x',y')$ i $dA' \longmapsto dL$, ara hem de restringir els graus de llibertat en $(x',y')$, ja que per la seva geometria es compleix $y' = \tan(\pi/6)x'$ per la placa superior i $y' = -\tan(\pi/6)x'$ per la inferior, també ja sigui d'això o per la simetria de la situació física veiem que $\phi_{sup}(x,y) = \phi_{inf}(x,-y)$. Per tant:

\begin{equation}
	\phi_{sup} (x,y) =\frac{1}{4\pi\varepsilon}\int_{0}^{\frac{\sqrt{3}}{2}L} \frac{q/L}{\sqrt{(x-x')^2+(y-\frac{\sqrt{3}}{2}x')^2}}dx' = \frac{1}{4\pi\varepsilon}\frac{\sqrt{3}}{2}\ln \left[\frac{\frac{4}{\sqrt{3}}\sqrt{(x^2+y^2)-L(\sqrt{3}x+y-L)}-\frac{2}{3}(3x+\sqrt{3}y-2\sqrt{3}L)}{\frac{4}{\sqrt{3}}\sqrt{x^2+y^2}-\frac{2}{3}(3x-\sqrt{3}y)}\right]
\end{equation}
\textcolor{red}{revisar otra vez que eso estè bien}
Finalment:

\begin{align}
	\phi(x,y) = \phi_q(x,y) + \phi_{sup}(x,y) + \phi_{sup}(x,-y) \\
	\vec{E}(x,y) = -\nabla\phi(x,y)
\end{align}
\textcolor{red}{alinear las expresiones para que se vea bonito}

\newpage
\subsection{Dades experimentals de la pràctica 1}
\label{an:a2}
A continuació mostrem els valors experimentals obtinguts en el transcurs de la pràctica 1.
\begin{table}[h]
	\centering
	\renewcommand{\arraystretch}{1.2}
	\caption{Taula amb els valors experimentals mesurats per tal de poder calcular la capacitat del condensador de plaques planoparal·leles.}
	\begin{tabular}{cccc}
		\toprule
		$\Delta l_i$ (m) & $\Delta r_i$ (m) & $\Delta V_i$ (V) & $\frac{\Delta V_i\Delta l_i}{\Delta r_i}$ (V)\\
		\midrule
		0.005 $\pm$ 0.001 & 0.019 $\pm$ 0.001 & 1.00 $\pm$ 0.01 & 0.26 $\pm$ 0.23 \\
		0.005 $\pm$ 0.001 & 0.024 $\pm$ 0.001 & 1.00 $\pm$ 0.01 & 0.21 $\pm$ 0.20 \\
		0.010 $\pm$ 0.001 & 0.038 $\pm$ 0.001 & 1.00 $\pm$ 0.01 & 0.26 $\pm$ 0.16 \\
		0.010 $\pm$ 0.001 & 0.050 $\pm$ 0.001 & 1.00 $\pm$ 0.01 & 0.20 $\pm$ 0.14 \\
		0.005 $\pm$ 0.001 & 0.057 $\pm$ 0.001 & 1.00 $\pm$ 0.01 & 0.09 $\pm$ 0.13 \\
		0.005 $\pm$ 0.001 & 0.063 $\pm$ 0.001 & 1.00 $\pm$ 0.01 & 0.08 $\pm$ 0.13 \\
		0.010 $\pm$ 0.001 & 0.070 $\pm$ 0.001 & 1.00 $\pm$ 0.01 & 0.14 $\pm$ 0.12 \\
		0.005 $\pm$ 0.001 & 0.071 $\pm$ 0.001 & 1.00 $\pm$ 0.01 & 0.07 $\pm$ 0.12 \\
		0.009 $\pm$ 0.001 & 0.066 $\pm$ 0.001 & 1.00 $\pm$ 0.01 & 0.14 $\pm$ 0.12 \\
		0.010 $\pm$ 0.001 & 0.034 $\pm$ 0.001 & 1.00 $\pm$ 0.01 & 0.29 $\pm$ 0.17 \\
		0.008 $\pm$ 0.001 & 0.024 $\pm$ 0.001 & 1.00 $\pm$ 0.01 & 0.33 $\pm$ 0.20 \\
		0.005 $\pm$ 0.001 & 0.014 $\pm$ 0.001 & 1.00 $\pm$ 0.01 & 0.36 $\pm$ 0.27 \\
		0.005 $\pm$ 0.001 & 0.009 $\pm$ 0.001 & 1.00 $\pm$ 0.01 & 0.56 $\pm$ 0.34 \\
		0.002 $\pm$ 0.001 & 0.004 $\pm$ 0.001 & 1.00 $\pm$ 0.01 & 0.50 $\pm$ 0.52 \\
		0.002 $\pm$ 0.001 & 0.003 $\pm$ 0.001 & 1.00 $\pm$ 0.01 & 0.67 $\pm$ 0.62 \\
		0.003 $\pm$ 0.001 & 0.004 $\pm$ 0.001 & 1.00 $\pm$ 0.01 & 0.75 $\pm$ 0.53 \\
		0.010 $\pm$ 0.001 & 0.005 $\pm$ 0.001 & 1.00 $\pm$ 0.01 & 2.00 $\pm$ 0.60 \\
		0.010 $\pm$ 0.001 & 0.006 $\pm$ 0.001 & 1.00 $\pm$ 0.01 & 1.67 $\pm$ 0.49 \\
		0.010 $\pm$ 0.001 & 0.006 $\pm$ 0.001 & 1.00 $\pm$ 0.01 & 1.67 $\pm$ 0.49 \\
		0.010 $\pm$ 0.001 & 0.006 $\pm$ 0.001 & 1.00 $\pm$ 0.01 & 1.67 $\pm$ 0.49 \\
		0.010 $\pm$ 0.001 & 0.005 $\pm$ 0.001 & 1.00 $\pm$ 0.01 & 2.00 $\pm$ 0.60 \\
		0.010 $\pm$ 0.001 & 0.005 $\pm$ 0.001 & 1.00 $\pm$ 0.01 & 2.00 $\pm$ 0.60 \\
		0.010 $\pm$ 0.001 & 0.006 $\pm$ 0.001 & 1.00 $\pm$ 0.01 & 1.67 $\pm$ 0.49 \\
		0.005 $\pm$ 0.001 & 0.005 $\pm$ 0.001 & 1.00 $\pm$ 0.01 & 1.00 $\pm$ 0.49 \\
		0.002 $\pm$ 0.001 & 0.005 $\pm$ 0.001 & 1.00 $\pm$ 0.01 & 0.40 $\pm$ 0.45 \\
		0.002 $\pm$ 0.001 & 0.006 $\pm$ 0.001 & 1.00 $\pm$ 0.01 & 0.33 $\pm$ 0.41 \\
		0.003 $\pm$ 0.001 & 0.007 $\pm$ 0.001 & 1.00 $\pm$ 0.01 & 0.43 $\pm$ 0.38 \\
		0.005 $\pm$ 0.001 & 0.011 $\pm$ 0.001 & 1.00 $\pm$ 0.01 & 0.45 $\pm$ 0.30 \\
		\bottomrule
	\end{tabular}
	\label{tab:valores}
\end{table}

Amb aquests valors i usant l'equació \eqref{eq1.17} podem determinar la capacitat del condensador de plaques planoparal·leles dibuixat.

\newpage
\subsection{Càlcul d'incerteses de la pràctica 1}
\label{an:a3}

\newpage
\subsection{Codis de les simulacions de la pràctica 1}
\label{an:a4}
Tot seguit adjuntem els diferents codis usats per generar les simulacions dels camps i les corbes equipotencials usant Python:.

El codi per la simulació del condensador de plaques planoparal·leles és:
\begin{lstlisting}
import numpy as np
import matplotlib.pyplot as plt

# Definim el camp electric 
def E(q, r0, x, y):
rx, ry = x - r0[0], y - r0[1]
dist = (rx**2 + ry**2)**1.5
return q * rx / dist, q * ry / dist

# Definim el potencial electric
def V(q, r0, x, y):
return q / np.hypot(x - r0[0], y - r0[1])

# Malla
x = np.linspace(-6, 6, 100)
y = np.linspace(-5, 5, 100)
X, Y = np.meshgrid(x, y)

q1, q2 = 1, -1
d = 4
charges = [(q1, (d/2, 0)), (q2, (-d/2, 0))]

Ex, Ey = np.zeros(X.shape), np.zeros(Y.shape)
Vt = np.zeros(X.shape)
for q, pos in charges:
ex, ey = E(q, pos, X, Y)
Ex += ex
Ey += ey
Vt += V(q, pos, X, Y)

fig, ax = plt.subplots()
ax.streamplot(X, Y, Ex, Ey, color=np.log(np.hypot(Ex, Ey)), linewidth=0.7, cmap=plt.cm.plasma, density=1)
ax.contour(X, Y, Vt, levels=np.linspace(-2, 2, 20), colors='black', linestyles='solid', linewidths=0.5)

for q, pos in charges:
color = 'red' if q < 0 else 'blue'
ax.scatter(*pos, color=color, s=100)

ax.set_aspect('equal')
ax.set_xticks([])
ax.set_yticks([])
ax.spines['top'].set_visible(True)
ax.spines['right'].set_visible(True)
ax.spines['left'].set_visible(True)
ax.spines['bottom'].set_visible(True)
plt.show()
\end{lstlisting}

El codi per la simulació dels dos fils infinits és:
\begin{lstlisting}
import numpy as np
import matplotlib.pyplot as plt

# Definim el camp electric
def E(q, r0, x, y):
    rx, ry = x - r0[0], y - r0[1]
    dist = (rx**2 + ry**2)**1.5
    return q * rx / dist, q * ry / dist

# Definim el potencial electric
def V(q, r0, x, y):
    return q / np.hypot(x - r0[0], y - r0[1])

# Malla
x = np.linspace(-6, 6, 100)
y = np.linspace(-5, 5, 100)
X, Y = np.meshgrid(x, y)

q1, q2 = 1, -1
d = 4
charges = [(q1, (d/2, 0)), (q2, (-d/2, 0))]

Ex, Ey = np.zeros(X.shape), np.zeros(Y.shape)
Vt = np.zeros(X.shape)
for q, pos in charges:
    ex, ey = E(q, pos, X, Y)
    Ex += ex
    Ey += ey
    Vt += V(q, pos, X, Y)

fig, ax = plt.subplots()
ax.streamplot(X, Y, Ex, Ey, color=np.log(np.hypot(Ex, Ey)), linewidth=0.7, cmap=plt.cm.plasma, density=1)
ax.contour(X, Y, Vt, levels=np.linspace(-2, 2, 20), colors='black', linestyles='solid', linewidths=0.5)

for q, pos in charges:
    color = 'red' if q < 0 else 'blue'
    ax.scatter(*pos, color=color, s=100)

ax.set_aspect('equal')
ax.set_xticks([])
ax.set_yticks([])
ax.spines['top'].set_visible(True)
ax.spines['right'].set_visible(True)
ax.spines['left'].set_visible(True)
ax.spines['bottom'].set_visible(True)
plt.show()
\end{lstlisting}

El codi per la simulació dels dos plans secants i el fil infinit és pel mètode de les càrregues imatge és:
\begin{lstlisting}
import numpy as np
import matplotlib.pyplot as plt
from matplotlib import colormaps
import sympy as sp
import latexify

q = 1
d= 4.603
nx, ny = 128, 128
L = (10.019+10.610)/2

#cosas de los graficos
x = np.linspace(-2.5, 13, 1500) #-1.5 / 4.2 / 50
y = np.linspace(-7, 7, 1500) #-2.5 / 2.5 / 50
X, Y = np.meshgrid(x, y)



#func potencaial
def V(x,y):
return -q*(1/np.sqrt((x-d)**2+(y)**2) - 1/np.sqrt((x-d/2)**2+(y-d*np.sqrt(3)/2)**2) +1/np.sqrt((x+d/2)**2+(y-d*np.sqrt(3)/2)**2) -1/np.sqrt((x+d)**2+(y)**2) +1/np.sqrt((x+d/2)**2+(y+d*np.sqrt(3)/2)**2) -1/np.sqrt((x-d/2)**2+(y+d*np.sqrt(3)/2)**2))

Z = V(X, Y)

def E_x(x,y):
return -(q*((x-d)/(np.sqrt((x-d)**2+(y)**2))**(3/2) - (x-d/2)/(np.sqrt((x-d/2)**2+(y-d*np.sqrt(3)/2)**2))**(3/2) +(x+d/2)/(np.sqrt((x+d/2)**2+(y-d*np.sqrt(3)/2)**2))**(3/2) -(x+d)/(np.sqrt((x+d)**2+(y)**2))**(3/2) +(x+d/2)/(np.sqrt((x+d/2)**2+(y+d*np.sqrt(3)/2)**2))**(3/2) -(x-d/2)/(np.sqrt((x-d/2)**2+(y+d*np.sqrt(3)/2)**2))**(3/2)))

def E_y(x,y):
return -(q*((y)/(np.sqrt((x-d)**2+(y)**2))**(3/2) - (y-d*np.sqrt(3)/2)/(np.sqrt((x-d/2)**2+(y-d*np.sqrt(3)/2)**2))**(3/2) +(y-d*np.sqrt(3)/2)/(np.sqrt((x+d/2)**2+(y-d*np.sqrt(3)/2)**2))**(3/2) -(y)/(np.sqrt((x+d)**2+(y)**2))**(3/2) +(y+d*np.sqrt(3)/2)/(np.sqrt((x+d/2)**2+(y+d*np.sqrt(3)/2)**2))**(3/2) -(y+d*np.sqrt(3)/2)/(np.sqrt((x-d/2)**2+(y+d*np.sqrt(3)/2)**2))**(3/2)))

E_xx = E_x(X,Y)
E_yy = E_y(X,Y)

#a por el plot
fig =plt.figure(figsize=(10.714, 9.286), label= 'Linies de camp') 
ax = fig.add_subplot(111)

ax.contour(X,Y,Z, levels =np.linspace(-3, 3, 30), colors='black',linewidths=0.5, linestyles='solid')
ax.streamplot(X,Y,E_xx,E_yy, density=2.5, color=np.log(np.hypot(E_xx, E_yy)), cmap='plasma', linewidth=0.5)
#plt.contourf(X, Y, normP, cmap="rainbow", levels=60)
#plt.contourf(X, Y, normPM, cmap="rainbow", levels=60)


#puntos de las cargas
#pa = np.linspace(0,L*np.sqrt(3)/2, 2000)

ax.plot([d],[0], 'bo', markersize=14)
ax.plot([d/2],[d*np.sqrt(3)/2], 'ro', markersize=14)
ax.plot([-d/2],[d*np.sqrt(3)/2], 'bo', markersize=14)
ax.plot([-d/2],[-d*np.sqrt(3)/2], 'bo', markersize=14)
ax.plot([d/2],[-d*np.sqrt(3)/2], 'ro', markersize=14)

#ax.plot([pa],[np.sqrt(3)/3*pa],'ro', markersize=3)
#configs del grafico
ax.set_xticks([])
ax.set_yticks([])
ax.set_aspect('equal')
plt.show()
\end{lstlisting}

El codi per la simulació analítica dels dos plans secants i el fil infinit és:
\begin{lstlisting}
import numpy as np
import matplotlib.pyplot as plt
from matplotlib import colormaps
import sympy as sp
import latexify



L = (10.019+10.610)/2
q =1
d = 4.603
u = (q/L)*np.sqrt(3)/2

def f(x, y):
a = 4/np.sqrt(3)
b = a*np.sqrt(x**2+y**2)-(2/3)*(3*x-np.sqrt(3)*y)
p = 1/np.sqrt((x-d)**2+y**2)
return u*np.log(((a*np.sqrt((x**2+y**2)+L*(-np.sqrt(3)*x-y+L))+2/3*(-3*x-np.sqrt(3)*y+2*np.sqrt(3)*L)))/(a*np.sqrt(x**2+y**2)-(2/3)*(3*x+np.sqrt(3)*y)))+u*np.log((a*np.sqrt((x**2+y**2)+L*(-np.sqrt(3)*x+y+L))+2/3*(-3*x+np.sqrt(3)*y+2*np.sqrt(3)*L))/b)-q*p

x = np.linspace(-2.5, 13, 1500) #-1.5 / 4.2 / 50
y = np.linspace(-7, 7, 1500) #-2.5 / 2.5 / 50
X, Y = np.meshgrid(x, y)
Z = f(X, Y)


#ojo esto esta de extra para probar, el streamplot creo que va bien con linspace, pero el quiver hay que darle aranges que sino....
#x = np.arange(-1.5,4.2,.5)
#y = np.arange(-2.5,2.5,.5) 

# Meshgrid 
X,Y = np.meshgrid(x,y)
#se acabo lo extra

def F(x,y):
return 4/np.sqrt(3)*np.sqrt(x**2+y**2+L*(y-np.sqrt(3)*x+L))+2/3*(np.sqrt(3)*y-3*x+2*np.sqrt(3)*L)

def F_x(x,y):
return 2/np.sqrt(3)*(2*x-np.sqrt(3)*L)/(np.sqrt(x**2+y**2+L*(y-np.sqrt(3)*x+L)))-2

def F_y(x,y):
return 2/np.sqrt(3)*(2*y+L)/(np.sqrt(x**2+y**2+L*(y-np.sqrt(3)*x+L)))+2/np.sqrt(3)


def V(x,y):
return 4/np.sqrt(3)*np.sqrt(x**2+y**2)+2/3*(np.sqrt(3)*y-3*x)

def V_x(x,y):
return 4/np.sqrt(3)*x/(np.sqrt(x**2+y**2))-2

def V_y(x,y):
return 4/np.sqrt(3)*y/np.sqrt(x**2+y**2)+2/np.sqrt(3)


Ex = -u*(F_x(X,Y)/F(X,Y) + F_x(X,-Y)/F(X,-Y) -V_x(X,Y)/V(X,Y)-V_x(X,-Y)/V(X,-Y)) -q*(X-d)/(((X-d)**2+Y**2)**(3/2))
Ey = -u*(F_y(X,Y)/F(X,Y) - F_y(X,-Y)/F(X,-Y) -V_y(X,Y)/V(X,Y)+V_y(X,-Y)/V(X,-Y)) -q*Y/(((X-d)**2+Y**2)**(3/2))

normE = np.log(np.sqrt(Ex**2+Ey**2))
normP = np.log(-Z)
normPM = np.log(Z)

fig =plt.figure(figsize=(10.714, 9.286), label= 'Linies de camp') 
ax = fig.add_subplot(111)

ax.contour(X,Y,Z, levels =np.linspace(-3, 3, 30), colors='black',linewidths=0.5, linestyles='solid')
ax.streamplot(X,Y,Ex,Ey, density=2.5, color=np.log(np.hypot(Ex, Ey)),cmap='plasma', linewidth=0.5)
#colorin = ax.contourf(X, Y, normE, cmap="rainbow", levels=60)
#plt.contourf(X, Y, normP, cmap="rainbow", levels=60)
#plt.contourf(X, Y, normPM, cmap="rainbow", levels=60)
ax.set_xticks([])
ax.set_yticks([])

ax.plot([d],[0], 'bo', markersize=14)
pa = np.linspace(0,L*np.sqrt(3)/2-.01, 2000)
ax.plot([pa],[np.sqrt(3)/3*pa],'ro', markersize=3)
ax.plot([pa],[-np.sqrt(3)/3*pa],'ro', markersize=3)

plt.show()
\end{lstlisting}

\textcolor{red}{modificar el codigo para que se vea bonito que está lleno de comentarios o cosas random}

\end{appendices}

\end{document}
